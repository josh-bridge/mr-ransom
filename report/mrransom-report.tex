\documentclass[12pt]{article}

\usepackage[utf8]{inputenc}
\usepackage[english]{babel}
\usepackage{natbib}
\usepackage[T1]{fontenc}
\usepackage{setspace}
\usepackage{graphicx}
\usepackage{hyperref}
\usepackage{csquotes}

\title{\vspace{3cm}\textbf{Information and Network Security}\\6G6Z1012}
\author{Joshua Michael Ephraim Bridge\\14032908\\joshua.m.bridge@stu.mmu.ac.uk}

\pagestyle{headings}

\begin{document}

\maketitle

\tableofcontents

% . Distribution scheme: How is the ransomware distributed to potential victims? What is the download and execution procedures?
% 2. Obfuscation: How the ransomware stay hidden during infection and operation to prevent removal and analysis.
% 3. Command and Control (C&C) Communications: Describe communications between the infected machine and the C&C server, e.g., key generation.
% 4. What files will it encrypt? Describe how to search for local and network drives? How to get files from the drive? and What are the supported file extensions?
% 5. Is there a way to decrypt the files or identify, detect and remove the ransomware?
% 6. What are the payment instructions like?

\newpage

\section{Ransomware review}
  \subsection{CryptoWall}
    CryptoWall was one of the largest ransomwares in recent years, with its highest popularity in 2014. Losses as a result of CryptoWall attacks in 2014 alone totalled around \$14 million \citep{fbi2014cryptowall}
  \subsection{Petya and Mischa}
    Ransomware-as-a-service (RaaS)

  \subsection{Locky}

  \subsection{Cerber}
    Ransomware-as-a-service (Raas)

\newpage

\section{Ransomware design}

\newpage

\bibliographystyle{agsm}
\bibliography{mrransom-report}

\end{document}
